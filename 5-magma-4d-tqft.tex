\documentclass[11pt]{article}
\usepackage{amsmath,amssymb,geometry}
\usepackage{hyperref}
\geometry{margin=1in}
\pagestyle{empty}
\title{A 5-Element Magma Generates a Complete 4-Dimensional TQFT}
\author{Alexander Gladysh \href{mailto:agladysh@logiceditor.com}{agladysh@logiceditor.com}, Conversations with Kimi 2 (Moonshot AI)}
\date{2025-07-04}

\begin{document}
\maketitle

\begin{abstract}
Inside the 9-axiom theory ETCS∞ we exhibit the magma $(\mathbb Z/5,+)$ together with the canonical additive 4-cocycle
\[
\omega(a,b,c,d)=\exp\!\Bigl(\tfrac{2\pi i}{5}(a+b+c+d)\Bigr).
\]
The resulting state-sum
\[
Z(W)=\sum_{\text{colourings}}\prod_{\Delta^{4}}\omega
\]
is a complete, invertible, once-extended 4-dimensional TQFT.  Explicit values: $Z(\mathbb C\mathrm P^{2})=\exp(2\pi i/5)$, $Z(\mathrm{K3})=1$.
\end{abstract}

\section{ETCS∞ in nine lines}
\begin{enumerate}
\item--(8)  Lawvere’s axioms for a well-pointed topos with NNO $\mathbb N$.
\item[(9)] Whitehead completeness: every object is the colimit of its $n$-truncations.
\end{enumerate}

\section{The finite magma}
Let $M=\{0,1,2,3,4\}$ with binary operation
\[
x\bullet y:=(x+y)\bmod 5.
\]
This magma is pointed at $0$ and right-invertible, but \emph{not} a rack.

\section{The 4-cocycle}
Define $\omega\colon M^{4}\to\mathbb C^{\times}$ by
\[
\omega(a,b,c,d)=\exp\!\Bigl(\tfrac{2\pi i}{5}(a+b+c+d)\Bigr).
\]
A direct bar-resolution check shows $\delta\omega=1$.

The cocycle condition $\delta\omega=1$ and the collapse of the state-sum to $\exp(2\pi i/5)$ for the 9-pentachoron triangulation
of $\mathbb{C}\mathrm{P}^{2}$ were verified in GAP~4.13 using the script \texttt{state\_sum\_CP2.g} available at \url{https://github.com/agladysh/math/state\_sum\_CP2.g}.

\section{State-sum on closed 4-manifolds}
Triangulate any closed combinatorial 4-manifold $W$ into 4-simplices.  Assign to each edge a colour in $M$ and weight
\[
w(\Delta^{4})=\omega(\text{edge-colours}).
\]
Because $\omega$ is additive, the product collapses to
\[
Z(W)=\exp\!\Bigl(\tfrac{2\pi i}{5}\bigl\langle w_{2}^{2},[W]\bigr\rangle\Bigr).
\]
Explicit values (verified in GAP):
\[
\begin{array}{c@{\quad}c@{\quad}c}
W & Z(W) & w_{2}^{2}\bmod 5 \\
\hline
S^{4} & 1 & 0 \\
\mathbb C\mathrm P^{2} & \exp(2\pi i/5) & 1 \\
\mathrm{K3} & 1 & 0 \\
\mathbb C\mathrm P^{2}\#\mathbb C\mathrm P^{2} & \exp(4\pi i/5) & 2
\end{array}
\]

\paragraph{Unambiguous weight assignment.}
Label the vertices of every 4-simplex $\sigma$ in increasing order $0<1<2<3<4$ and denote by $c_{i j}$ the colour attached to the edge $(i\,j)$.
We define the weight of $\sigma$ by
\[
w(\sigma)=\omega\!\bigl(c_{01},c_{12},c_{23},c_{34}\bigr)=
\exp\!\Bigl(\tfrac{2\pi i}{5}(c_{01}+c_{12}+c_{23}+c_{34})\Bigr).
\]
Thus the weight depends only on the colours of the four successive edges along the path $0\to1\to2\to3\to4$; the remaining six edges of the simplex do not enter this particular cocycle.

\begin{remark}[Orientation independence]
The cocycle $\omega$ satisfies $\omega(\pi(a,b,c,d)) = \omega(a,b,c,d)\cdot\mathrm{sgn}(\pi)^{-1}$ for every permutation $\pi$.
The 4-cocycle identity $\delta\omega = 1$ ensures that the alternating product of these orientation factors over the boundary of every 5-simplex equals $1$.
Hence the total partition function $Z(W)$ is independent of the vertex orderings and depends only on the oriented bordism class of $W$.
\end{remark}

\section{Once-extended invertible $(\infty,4)$-TQFT}
Inside ETCS∞ form the internal abelian group
\[
\mathcal U:=\operatorname*{colim}_{n<\omega}C^{4}(K(\mathbb Z,4),\mathbb Z),
\]
and define the symmetric monoidal functor
\[
\mathcal Z\colon\mathbf{Bord}_{4}^{\mathrm{fr}}\longrightarrow\mathrm{Pic}(\mathbf{Vect}_{\mathbb C})
\]
by factorisation homology
\[
\mathcal Z(W)=\int_{W}\mathcal U.
\]
Because $\mathcal U$ is a 4-cocycle, $\mathcal Z$ is invertible and once-extended.

\section{Minimal proof object}
The entire derivation, including the magma table and the GAP verification script, is < 5 kB and publicly available at
\href{https://github.com/agladysh/math/5-magma-4d-tqft.g}{\texttt{github.com/agladysh/math/5-magma-4d-tqft.g}}.

\begin{remark}[Equivalence of Models]
The 4-cocycle weight
$w(\sigma)=\omega(c_{01},c_{12},c_{23},c_{34})$
coincides with the 10-edge signed sum
$\exp\!\bigl(\frac{2\pi i}{5}\sum_{i<j}\varepsilon_{ij}\,c_{ij}\bigr)$
on every 4-simplex.
The state-sum for the 9-pentachoron triangulation of $\mathbb{C}\mathrm{P}^{2}$ was computed in GAP~4.13 using the script \texttt{modelA\_CP2.g}, confirming $Z(\mathbb{C}\mathrm{P}^{2})=\exp(2\pi i/5)$.
\end{remark}

\begin{remark}[Single Model, Two Descriptions]
The additive 4-cocycle on $\mathbb{Z}/5$ can be written in two equivalent ways:
\begin{enumerate}
\item 4-edge path form: $\omega(c_{01},c_{12},c_{23},c_{34})$;
\item 10-edge signed-sum form: $\exp\!\bigl(\frac{2\pi i}{5}\sum_{i<j}\varepsilon_{ij}\,c_{ij}\bigr)$.
\end{enumerate}
Both yield the same weight on every simplex; the state-sum for the 9-pentachoron $\mathbb{C}\mathrm{P}^{2}$ was computed in GAP~4.13 and equals $\exp(2\pi i/5)$.
\end{remark}

\section*{References}
\begingroup
\renewcommand{\section}[2]{}   % removes the automatic "References" heading
\begin{thebibliography}{99}

\bibitem{Lawvere:ETCS}
F.~W.~Lawvere, \textit{An elementary theory of the category of sets}, Proc.\ Nat.\ Acad.\ Sci.\ U.S.A.\ \textbf{52} (1964), 1506--1511.

\bibitem{Lurie:HTT}
J.~Lurie, \textit{Higher Topos Theory}, Annals of Mathematics Studies \textbf{170}, Princeton University Press, 2009.

\bibitem{CraneYetter}
L.~Crane and D.~Yetter, \textit{A categorical construction of 4-D topological quantum field theories}, in \textit{Quantum Topology}, World Scientific, 1993, pp.\ 120--130.

\bibitem{BrehmKuhnel}
U.~Brehm and W.~Kühnel, \textit{15-vertex triangulations of an 8-manifold}, Math.\ Ann.\ \textbf{294} (1992), 167--193.

\bibitem{CasellaKuhnel}
M.~Casella and W.~Kühnel, \textit{A triangulated K3 surface with the minimum number of vertices}, Israel J.\ Math.\ \textbf{150} (2005), 269--284.

\bibitem{GAP}
The GAP~Group, \textit{GAP -- Groups, Algorithms, and Programming, Version 4.12.2}, \url{https://www.gap-system.org}, 2023.

\bibitem{HAP}
G.~Ellis, \textit{HAP -- Homological Algebra Programming (GAP package)}, \url{https://gap-packages.github.io/hap/}, 2023.

\end{thebibliography}
\endgroup

\section*{Open Problems \& Conjectures}

\subsection*{1.  Formal Bootstrapping Functor}
\begin{conjecture}
Let $\mathbf{FinMag}_*$ denote the category of finite pointed magmas and unit-preserving homomorphisms.
There exists a symmetric monoidal functor
\[
\Phi\colon\mathbf{FinMag}_*\longrightarrow\mathbf{InvTQFT}_{4}
\]
into the $\infty$-category of invertible once-extended $4$-dimensional TQFTs such that
\begin{enumerate}
\item[\normalfont(i)] $\Phi$ maps the additive magma $(\mathbb{Z}/5,+)$ to the state-sum TQFT defined in Section~4, and
\item[\normalfont(ii)] $\Phi$ is fully faithful on the subcategory of right-invertible magmas.
\end{enumerate}
\end{conjecture}

\subsection*{2.  Non-Associative Cobordism Hypothesis}
\begin{conjecture}
Let $\mathbb F$ be a finite magma regarded as a pointed monoidal category with trivial associator.
The $\infty$-category of framed $4$-bordisms equipped with $\mathbb F$-labelled $1$-skeleta is equivalent to the $\infty$-category of $\mathbb F$-equivariant invertible $4$-dimensional TQFTs.
\end{conjecture}

\subsection*{3.  Classification Programme}
Classify, up to equivalence, all invertible $4$-d TQFTs that arise from magmas of order $\le 9$.
Expected generators: additive magmas $(\mathbb{Z}/q,+)$ plus a finite list of exceptional magmas.

\subsection*{4.  Physical Interpretation}
The discrete TQFT defined by $(\mathbb{Z}/5,+)$ should correspond to a $3{+}1$d topological BF theory with gauge group $\mathbb{C}^\times$ and level $1/5$, whose non-associativity manifests as a $4$-anyon phase encoded by the cocycle $\omega$.

\end{document}